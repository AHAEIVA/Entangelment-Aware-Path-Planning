\section{Introduction}
\label{sec:introduction}
As part of industrial 4.0 and the need for flexible and efficient manufacturing, transportation and logistics, companies increasingly rely on autonomous mobile robots to handle and distribute resources. A key component to successfully implement applications of large scale robots is to handle collision-free multi-agent path planning and control the flow of robots in the system. Many research efforts have been devoted to this field in recent years from different perspectives.
\smallbreak
% Decentralised vs centralised
A multi-agent system can generally be categorised into two architectures: a centralised or decentralised system. In a centralised system, all robots transmit information to a centralised ground station. The ground station processes all the robots intentions and sends a collision-free path back to the robot \cite{CBS_paper}. While centralised systems have many benefits, in practice, centralised systems heavily depend on powerful machines and a fast communication network, which is costly to be deployed in large scale environments with a large number of robots.
In a decentralised system, each robot computes its own path through the environment, creating a more scalable solution compared to the centralised system. However, the problem is the complexity of predicting the flow of the environment as a whole, as each robot might not have access to the entire state space of the system.
\medbreak
% Local planning and global planning
For many decades, researchers have developed many different ideas on how to solve the decentralised collision-free path for a robot with incomplete environment information. Many papers have succeeded in optimising this\cite{Mapper}\cite{MAG_planner}, but these methods only optimise the flow on a very local level, without considering the flow on a global level. In this paper, we focus on optimising a decentralised global planner, considering not only the distance of the planned path but also the traffic flow throughout the planned time. Inspired by the state-of-the-art estimated time of arrival method \cite{GOOGLE_ETA}, we utilise \ac{GNN} for estimating the flow of the system and predict better trajectories than the naive shortest route. We will concentrate on the decentralised system, where limited communication is available, and we do not have perfect information about the other robots. The solution needs to predict the movement of other robots in the system in order to maximise the flow. 
\begin{figure}[t]
    \centering
    \includesvg[width=0.49\textwidth]{images/introduction/Method_overview.svg}
    \caption{Method overview: Explanation is coming up}
    \label{fig:overview}
\end{figure}

\textcolor{red}{Contributions}
\noindent The major contributions of this study are the followings:
\begin{itemize}
    \item First, we investigate the global multi-agent path planning problem in highly active industrial environments. To model these environments and locate the potential bottle-neck areas, we propose an automatic bottle-neck detection and topological graph construction method from 2d occupancy grid maps.
    \item Secondly, we propose a decentralized global planner utilizing graph neural networks for multi-robot environments for optimizing flow in industrial scenarios.
    \item Finally,  
\end{itemize}
The rest of this paper is organised as follows. 
Section \ref{sec:related_work} introduces the state-of-the-art in the field of multi-agent path-planning and global path planning. Section \ref{sec:methodology} provides some preliminaries for our problem formulation along with the approach of our \ac{DMAGGPP} for generalised indoor industrial environments. Section \ref{sec:experiments} explains the experiment setup followed by the results of \ac{DMAGGPP} in various grid world environments which is presented in section \ref{sec:results}. Finally, we give a brief conclusion in section \ref{sec:conclusion}.



