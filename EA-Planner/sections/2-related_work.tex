\section{State of the art}
\label{sec:related_work}

% Paragraph 1: Introduction to Tether Importance and Entanglement Definition
The risk of entanglement with obstacles presents a serious challenge for robots with conventional path planners. To systematically address this, a taxonomy of entanglement definitions was presented in \cite{definitions}, where existing interpretations from the literature were cataloged and new definitions were introduced, paving the way for developing new path planning strategies that account for entanglement.

% Paragraph 2: Early Approaches - 2D Offline Path Planning (Single Robot)
Initial efforts to develop entanglement-aware path planners were often focused on two-dimensional environments and employed offline computation strategies. For instance, in \cite{rov_mccammon} and \cite{mechsy2017novel}, methods were proposed for planning paths in 2D to cover a predefined set of waypoints while considering tether constraints. Notably, in \cite{mechsy2017novel}, the path planning problem for a \ac{ROV} was formulated as a mixed-integer programming problem. In that approach, a \ac{TSP} was first solved to find an optimal waypoint sequence, and then homotopic constraints were incorporated during path generation to minimize the likelihood of tether entanglement. While effective for predefined scenarios, these offline methods lack the adaptability required for dynamic environments. A more recent approach to the 2D coverage path planning problem for tethered robots was presented in \cite{peng2025spanning}, where spanning tree-based offline optimization was employed.

% Paragraph 3: Advancements - 2D Online Path Planning (Single Robot)
To address the need for real-time adaptability, subsequent research explored online path planning algorithms, still primarily in 2D. In \cite{kim}, a homotopy-augmented topological approach combined with graph search techniques was introduced, allowing for dynamic adjustments to the path based on environmental perception. Similarly, in \cite{withy}, a hybrid A* variant utilizing a modified tangent graph was developed. This method efficiently plans curvature-constrained paths for tethered robots subject to winding angle constraints, demonstrating guarantees and providing simulation results for online entanglement avoidance.

% Paragraph 4: Scaling Complexity - Multi-Robot Systems in 2D
The complexity of tether management increases significantly when coordinating multiple robots. Foundational work in this area was laid in \cite{hert1996ties}, followed by methods proposed in \cite{zhang2019planning}. More recently, in \cite{cao2023neptune}, an efficient online path planner for multi-robot systems was presented. In this method, a homotopy-based high-level planner was integrated with trajectory optimization and smoothing techniques to generate entanglement-free paths. However, despite its online capability, this approach remains constrained to 2D environments.

The online planners described above represent a significant step towards real-time tether management; however, they are limited to 2D environments. Moreover, while preventive paths to avoid entanglement can be planned, strategies for path planning once tether entanglement has already occurred are not provided.

% Paragraph 5: Moving to Three Dimensions - 3D Path Planning (Single Robot)
Real-world applications frequently require navigation in 3D environments, such as with underwater robots. Consequently, in \cite{bhattacharya2012topological} and \cite{martinez2021optimization}, topological aspects and optimization techniques for 3D tethered navigation were explored. In \cite{petit2022tape}, a 3D exploration path planner incorporating explicit contact avoidance constraints for the tether was presented, facilitating safer navigation for single tethered robots in complex three-dimensional spaces.

% Paragraph 6: 3D Multi-Robot Path Planning
The increased complexity of 3D multi-robot scenarios was addressed in \cite{hert1999motion}, where earlier 2D work was extended to three dimensions. Further advancements were introduced in \cite{patil2023coordinating} and \cite{cao2023path}, where path planning strategies explicitly considering the topological constraints imposed by multiple interacting tethers in 3D were proposed. While these methods advance the state of the art in multi-robot coordination, they are generally designed for offline computation and are not suited for online path planning where real-time implementation is essential.

% Paragraph 7: Identifying the Research Gap
In summary, existing path planners that account for tether constraints often face limitations for practical online \ac{CPP} in complex 3D settings. Many are too computationally intensive for real-time use \cite{mechsy2017novel, hert1999motion, patil2023coordinating, cao2023path}, lack integrated tether-aware CPP frameworks, or rely on simplifying assumptions such as 2D environments or basic obstacle shapes \cite{kim, withy, cao2023neptune}, hindering generalization to real-world inspection tasks.

% Paragraph 8: Introducing the Proposed Solution (REACT) based on its key contributions
To address these limitations, a novel approach named \ac{REACT} is proposed, which enables real-time, entanglement-aware path planning in arbitrary 3D environments.